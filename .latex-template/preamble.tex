\usepackage[top=25truemm,bottom=25truemm,left=25truemm,right=25truemm]{geometry}
\usepackage[dvipdfmx]{graphicx,hyperref,color}
\usepackage{amsmath,amssymb,braket,bm,mathtools,amsfonts,url, cancel, mathrsfs}
\usepackage{ascmac}% to use itembox
\usepackage{algorithm,algorithmic}
%\usepackage[cache=false]{minted}
\usepackage{physics}
\usepackage{tikz}% quantum circuit
\usepackage{comment}
\usepackage{here}
\usepackage{xcolor}
\hypersetup{
    colorlinks=true,
    linkcolor=black
}

% number figures, tables and equations within the sections
\numberwithin{equation}{section}
%\numberwithin{equation}{subsection}
\numberwithin{figure}{section}
\numberwithin{table}{section}
% Omit `.0` in equation numbers for non-existent subsections.
% \renewcommand*{\theequation}{%
%   \ifnum\value{subsection}=0 %
%     \thesection
%   \else
%     \thesubsection
%   \fi
%   .\arabic{equation}%
% }

\renewcommand{\baselinestretch}{1.25}

\newtheoremstyle{mystyle}%   % スタイル名
    {3pt}%b                  % 上部スペース
    {3pt}%                   % 下部スペース
    {}%                      % 本文フォント
    {12pt}%                      % 1行目のインデント量
    {\bfseries}%             % 見出しフォント
    {.}%                     % 見出し後の句読点
    {\newline}%              % 見出し後のスペース
    {\thmname{#1}\thmnumber{ #2}\thmnote{ #3}}

\theoremstyle{mystyle}
\newtheorem{definition}{Definition}[section]
\newtheorem*{definition*}{Definition}
\newtheorem{theorem}{Theorem}[section]
\newtheorem*{theorem*}{Theorem}
\newtheorem{corollary}{Corollary}[theorem]
\newtheorem*{corollary*}{Corollary}
\newtheorem{lemma}[theorem]{Lemma}
\newtheorem*{lemma*}{Lemma}


% Latin abbreviations
\newcommand{\etal}{\textit{et al.}}
\newcommand{\eg}{\textit{e.g.}}
\newcommand{\cf}{\textit{c.f.}}
\newcommand{\ie}{\textit{i.e.}}
\newcommand{\etc}{\textit{etc.}}

% log type
\let\Re\relax
\DeclareMathOperator{\Re}{Re}
\let\Im\relax
\DeclareMathOperator{\Im}{Im}
\DeclareMathOperator{\Ker}{Ker}
\DeclareMathOperator{\Image}{Im}
\DeclareMathOperator{\Supp}{Supp}
\DeclareMathOperator{\sgn}{sgn}
\DeclareMathOperator{\sign}{sign}
\DeclareMathOperator{\tr}{tr}
\DeclareMathOperator{\Tr}{Tr}
\DeclareMathOperator{\rank}{rank}
\DeclareMathOperator{\rk}{rk}
\DeclareMathOperator{\diag}{diag}
\DeclareMathOperator{\Var}{Var}
\DeclareMathOperator{\Cov}{Cov}

\DeclareMathOperator{\sech}{sech}
\DeclareMathOperator{\csch}{csch}
\DeclareMathOperator{\arcsec}{arcsec}
\DeclareMathOperator{\arccot}{arccot}
\DeclareMathOperator{\arccsc}{arccsc}
\DeclareMathOperator{\arccosh}{arccosh}
\DeclareMathOperator{\arcsinh}{arcsinh}
\DeclareMathOperator{\arctanh}{arctanh}
\DeclareMathOperator{\arcsech}{arcsech}
\DeclareMathOperator{\arccsch}{arccsch}
\DeclareMathOperator{\arccoth}{arccoth}

\DeclareMathOperator{\grad}{grad}
\DeclareMathOperator{\div}{div}
\DeclareMathOperator{\rot}{rot}
% limit type
\DeclareMathOperator*{\argmin}{arg~min}
\DeclareMathOperator*{\argmax}{arg~max}

% \mathcal
\newcommand{\calA}{\mathcal{A}}
\newcommand{\calB}{\mathcal{B}}
\newcommand{\calC}{\mathcal{C}}
\newcommand{\calD}{\mathcal{D}}
\newcommand{\calE}{\mathcal{E}}
\newcommand{\calF}{\mathcal{F}}
\newcommand{\calG}{\mathcal{G}}
\newcommand{\calH}{\mathcal{H}}
\newcommand{\calI}{\mathcal{I}}
\newcommand{\calJ}{\mathcal{J}}
\newcommand{\calK}{\mathcal{K}}
\newcommand{\calL}{\mathcal{L}}
\newcommand{\calM}{\mathcal{M}}
\newcommand{\calN}{\mathcal{N}}
\newcommand{\calO}{\mathcal{O}}
\newcommand{\calP}{\mathcal{P}}
\newcommand{\calQ}{\mathcal{Q}}
\newcommand{\calR}{\mathcal{R}}
\newcommand{\calS}{\mathcal{S}}
\newcommand{\calT}{\mathcal{T}}
\newcommand{\calU}{\mathcal{U}}
\newcommand{\calV}{\mathcal{V}}
\newcommand{\calW}{\mathcal{W}}
\newcommand{\calX}{\mathcal{X}}
\newcommand{\calY}{\mathcal{Y}}
\newcommand{\calZ}{\mathcal{Z}}

% mathbb
\newcommand{\bbN}{\mathbb{N}}
\newcommand{\bbZ}{\mathbb{Z}}
\newcommand{\bbQ}{\mathbb{Q}}
\newcommand{\bbR}{\mathbb{R}}
\newcommand{\bbC}{\mathbb{C}}

\newcommand{\bbE}{\mathbb{E}}
\newcommand{\bbV}{\mathbb{V}}

% mathrm
\newcommand{\NOT}{\mathrm{NOT}}
\newcommand{\AND}{\mathrm{AND}}
\newcommand{\NAND}{\mathrm{NAND}}
\newcommand{\OR}{\mathrm{OR}}
\newcommand{\XOR}{\mathrm{XOR}}
\newcommand{\CNOT}{\mathrm{CNOT}}
\newcommand{\Tof}{\mathrm{Toffoli}}

% matrix
\newcommand{\bmat}[1]{\begin{bmatrix}#1\end{bmatrix}}
\newcommand{\pmat}[1]{\begin{pmatrix}#1\end{pmatrix}}

% sin cos
\newcommand{\sif}[2]{\sin\qty(\frac{#1}{#2})}
\newcommand{\cof}[2]{\cos\qty(\frac{#1}{#2})}
\newcommand{\siff}[3]{\sin^{#1}\qty(\frac{#2}{#3})}
\newcommand{\coff}[3]{\cos^{#1}\qty(\frac{#2}{#3})}

\newcommand{\paulix}{\begin{pmatrix} 0 & 1 \\ 1 & 0 \end{pmatrix} }
\newcommand{\pauliy}{\begin{pmatrix} 0 & -i \\ i & 0 \end{pmatrix} }
\newcommand{\pauliz}{\begin{pmatrix} 1 & 0 \\ 0 & -1 \end{pmatrix} }

\newcommand{\rx}[1]{ \cos \frac{#1}{2} & -\sin\frac{#1}{2}  \\ \sin \frac{#1}{2}  & \cos\frac{#1}{2}}
\newcommand{\ry}[1]{ \cos \frac{#1}{2} & -i\sin\frac{#1}{2} \\ i\sin \frac{#1}{2} & \cos\frac{#1}{2}}
\newcommand{\rz}[1]{ \exp -i{#1}/2      & 0                  \\ 0                 & \exp i{#1}/2    }
\newcommand{\rot}[1]{ \cos {#1} & -\sin {#1} \\ \sin {#1} & \cos {#1}}

% others
\newcommand{\pure}[1]{|{#1}\rangle\langle{#1}|}
\newcommand{\combi}[2]{{}_{#1}\text{C}_{#2}}

% ref; https://github.com/SSoelvsten/latex-preamble-and-examples/tree/main/documents
